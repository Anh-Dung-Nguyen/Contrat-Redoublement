Nous avons donc réalisé le travail demandé :
\begin{itemize}
    \item Générer un contrat vierge
    \item Générer un contrat avec des notes
    \item Générer l'ensemble des contrats des redoublants avec leurs notes
    \item Générer un bilan
\end{itemize}


\noindent De plus, nous avons réalisé une interface graphique permettant : 
\begin{itemize}
  \item De générer le contrat d'un redoublant en cliquant sur le bouton « Génération du contrat de l'étudiant sélectionné »
  \item De générer le fichier bilan en cliquant sur le bouton ...
  \item De générer l'ensemble des contrats en cliquant sur le bouton « Génération de l'ensemble des contrats des redoublants »
  \item D'ajouter des croix dans la colonne « EC validé mais repassé en 2025-2026 » du contrat (ceci est un ajout au cahier des charges initial, il a été proposé par notre groupe afin de faciliter l'utilisation de NoteR)
\end{itemize}

Point important : d'une année à l'autre la maquette peut évoluer. Elle est décrite dans un fichier d'entrée \texttt{MCC} permettant à notre code de s'adapter automatiquement à celle-ci.